\subsubsection*{版本号\+:v1.\+0.\+0}

\paragraph*{2022年4月22日}

真实数据验证基本通过

\subsubsection*{版本号\+:v1.\+0.\+1}

\paragraph*{2022年4月22日}

从setting中抽出配置到yaml文件中

\subsubsection*{版本号\+:v1.\+0.\+2}

\paragraph*{2022年4月26日}

优化设置 实现不均匀划分 Jxyz文件变为完全不必须(不生成也能运行) 引入矩阵求解算法(\+Gij) 对目标函数和jacbi增加缓存显著提高这部分速度(这块速度约4倍)

\subsubsection*{版本号\+:v1.\+0.\+3}

\paragraph*{2022年4月28日}

纠正考虑没有射线穿过格子平滑度场景下,方程组构建时,由于n选择不当造成的异常

\subsubsection*{版本号\+:v1.\+0.\+4}

\paragraph*{2022年5月4日}

支持打包运行: 安装\+Cpython,pyinstaller包 配置\+Main.\+spec 命令运行 pyinstaller -\/D ..py 将configuration.\+yml放入\+Main.\+exe目录下 终端执行 不支持plt,所以这部分打包前必须取消引用 重构代码避免实现单例的时候重复调用init {\itshape 并发\+L\+B\+F\+G\+SB,但是会出现重复导入,运行速度理论最快达到n倍,n为计算机cpu核心数量,此方案似乎可行,但是几乎需要对代码进行非常大的修改}(当前代码可以)

\subsubsection*{版本号\+:v1.\+0.\+5}

\paragraph*{2022年5月5日}

对结果进行简单的平滑处理\+\_\+\+\_\+暂时不做优化,只针对马面 修正空气标记的一点问题 refs自动生成--会影响最终结果--需要分析原因 目标函数改为p范数---l1-\/l2 纠正了最后优化结果算法的 对结果进行处理的时候重新检查城墙--不应该有,但是ref有问题暂时先做优化

\subsubsection*{版本号\+:v1.\+0.\+6}

\paragraph*{2022年5月15日}

实现通过连续的xyz坐标得到j的工具,该工具被集成到\+Mesh\+\_\+tool中 完善p范数,使用二范数时候的雅可比来近似p范数的雅可比 通过引入setting.\+ini实现配置文件configuration.\+yml的替换,取消小工具中的配置文件 小幅度修改了β寻找功能 生成各种uml图 注释完善

\subsubsection*{版本号\+:v1.\+0.\+7}

\paragraph*{2022年6月1日}

修改日志记录类,更加像logging。 尝试消除探测器附近格子的密度过低问题,使用收紧约束简单加权重等方式,最终结果有待验证

\subsubsection*{版本号\+:v1.\+0.\+8}

\paragraph*{2022年7月30日}

论文仿真 分层密度曲线,模型相似度

\subsubsection*{版本号\+:v1.\+1.\+0}

\paragraph*{2022年7月30日}

支持不均匀划分的空间标记,8个点的位置和顺序如图所示\+:



融合python工具箱,支持性能监视器,结果的图像展示设置为选择项

标记夯土(城墙)--进而得到墙砖的格子 通过正演的射线与topo的交点信息来标记各个部分(城墙/空气) 降维算法\+\_\+\+\_\+有一定节省内存和不必要的效果,但是算法本身成本也很高--默认不适用已在\+Data\+Manager注释

\subsubsection*{版本号\+:v1.\+1.\+1}

\paragraph*{2022年10月15日}

实现反演接口——抛弃\+Setting.\+ini文件

改进seed算法使用带有平滑性的目标函数,使用线性搜索法作为优化器

四个接口to 刘国瑞\+: 三维坐标转换,获取成像区域最下边格子的物理坐标,根据交点信息生成refs和bounds文件,根据配置文件yaml\+\_\+file启动一个新的反演程序

生成mesh文件的接口to国睿

\subsubsection*{版本号\+:v1.\+1.\+2}

\paragraph*{2022年11月8日}

seed算法调整代码结构,图像融合实现从csv进行小波变换并显示图像

完全替换ryj代码

二维结果显示工具

紧急修复遗留问题\+\_\+\+\_\+\+Calcsensitivity替换后部分接口没有做同步修改

自动删除探测器信息缓存,避免重复调用造成的数据错误,但是不太支持多进程,如果需要请保证全局的探测器id和探测器编号对应。

\subsubsection*{版本号\+:v1.\+1.\+3}

\paragraph*{2022年12月25日}

seed算法重构实现块坐标下降方式,可选择优化算法

实现\+M\+N\+I求解算法 